{\bf
  {\LARGE
    \centerline{Memorándum de Entendimiento}
  }
  {\Large
    \centerline{entre la}
    \centerline{\institution}
    \centerline{y la}
    \centerline{Colaboración LAGO}
    \vspace*{0.2cm}
    \centerline{\datesignes}
  }
}

\section{Introducción}

Este Memorándum de Entendimiento describe los compromisos de los miembros de
la \institution~con el Proyecto LAGO. El propósito de este proyecto es el
diseño, la instalación, la puesta en marcha y la operación del Proyecto LAGO,
un observatorio extendido de Astropartículas a escala global. Sus principales
objetivos científicos se centran principalmente en tres áreas de la
investigación básica en Astropartículas: el Universo Extremo, fenomenología de
Meteorología Espacial, y Radiación Atmosférica a nivel del suelo. 

La red de detección LAGO está formada por detectores de partículas,
individuales o formando pequeños arreglos, ubicados a nivel del suelo e
instalados en diferentes sitios. La red abarca una gran distribución en latitudes
(actualmente desde México hasta la región Antártica), y altitudes (desde el
nivel del mar hasta más de 5000 metros sobre el nivel del mar), cubriendo un
extenso rango de rigideces de corte geomagnético y niveles de absorción y
reacción atmosféricos.

El proyecto LAGO es operado por la Colaboración LAGO, una red no centralizada,
distribuida y colaborativa integrada por investigadores y estudiantes de varias
instituciones de los diferentes países miembros del proyecto LAGO.

Este Memorándum de Entendimiento describe las contribuciones de largo plazo de
la \institution~al diseño, la construcción, la puesta en marcha y la operación
del proyecto LAGO. Se entiende que las contribuciones previstas de la
\institution~pueden ser modificadas o pueden ser agregadas responsabilidades
adicionales a las aquí descriptas.

Este Memorándum de Entendimiento se realiza entre la \institution, representada
aquí por \ifirg el \else la \fi \instrep, el representante LAGO de \country,
representado aquí por \ifcrg el \else la \fi \countryrep, y la Colaboración
LAGO, representada aquí por su Investigador Principal \lagopi, de aquí en
adelante referidos en conjunto como ``las Partes'', o separadamente como una
``parte''.

Este Memorándum no constituye una obligación contractual legal para ninguna de
las partes. Este documento refleja simplemente un arreglo que es actualmente
satisfactorio para las partes, quienes acuerdan a negociar a enmiendas
adicionales a este memorándum según se requiera para cumplir la evolución de
los requerimientos de la \institution~y/o el Proyecto LAGO.

\section{Personal}

\subsection{Listado del Personal Científico}

A continuación, se expone la lista completa de personal científico participante
(incluyendo personal técnico, posdocs y estudiantes) y la fracción de su tiempo
que estará destinada al proyecto durante el período de asociación cubierto por
este Memorándum y sus enmiendas:

\begin{center}
  \begin{tabular}{|p{5.5cm}|p{7.2cm}|p{2.4cm}|}
\hline
\ifes Nombre \fi
\ifen Name \fi 
\ifpt Nome \fi &
\ifes Posición \fi
\ifen Position \fi 
\ifpt NNN \fi & 
\ifes Compromiso \fi 
\ifen Commitment \fi 
\ifpt NNN \fi
\\
\hline
Alfredo Vega & Profesor Investigador & 0.2 \\
Iván Gonzalez & Profesor Investigador & 0.2 \\
José Villanueva & Profesor Investigador & 0.2 \\
Victor Cárdenas & Profesor Investigador & 0.2 \\
Graeme CAndlish & Profesor Investigador & 0.2 \\
Osvaldo Herrera & Profesor Investigador & 0.2 \\
Juan Magaña & Postdoc & 0.2 \\
Daniel Manriquez & Pregrado & 0 \\
\hline
FTE & -- & 1.4 \\
\hline
\end{tabular}
\end{center}


De acuerdo a las políticas de membresía de LAGO todo el personal de LAGO, de
ahora en adelante referidos como ``los miembros de la Colaboración LAGO
asociados a la \institution'' o simplemente ``los miembros LAGO de la
\institution'', participarán en al menos uno de los Grupos de Trabajo del
Proyecto LAGO. De esta forma, los miembros LAGO de la \institution~serán parte
de la lista de autores de la Colaboración LAGO.

El Representante de la Institución (ver sub-sección \ref{ires}) de la
\institution~comunicará mensualmente todas las incorporaciones, alejamientos o
cambios en el personal asociado al proyecto LAGO al Representante LAGO de
\country.

\subsection{Representante de la Institución\label{ires}}

\ifirg El \else La \fi \instrep~es
\ifirg el \else la \fi representante actual de la
\institution~frente a la Colaboración LAGO, y fue elegid\ifirg o \else a \fi por los miembros de
LAGO de la \institution~en un todo de acuerdo a las políticas de la
Colaboración LAGO. \ifirg El \else La \fi \instrep~actuará como el
contacto principal entre los miembros LAGO de la \institution~y la Colaboración
LAGO a través del representante LAGO de \country.

\section{Responsabilidades de Diseño, Fabricación, Puesta en Marcha y
Operación}

\subsection{Responsabilidades de Diseño y Fabricación Durante el Período de
Construcción}

\subsubsection{Descripción de los Ítems provistos por la \institution}

La \institution~está de acuerdo en comprometer su mejor esfuerzo para la
provisión de los siguientes componentes, materiales y servicios: 

\begin{center}
\begin{tabular}{|p{1.0cm}|p{1.5cm}|p{12.7cm}|}
\hline
\ifes Ítem \fi
\ifen Item \fi
\ifpt Item \fi
&
\ifes Cantidad \fi
\ifen Quantity \fi
\ifpt Cantidad \fi
&
\ifes Descripción \fi
\ifen Description \fi
\ifpt Descripcion \fi
\\
\hline
	\ifes 1 & -- & Materiales y componentes para el armado de un detector completo, excluyendo el PMT, la digitalizadora LAGO y la placa de desarrollo Nexys 2 (ver tabla 3) \\ \fi
	\ifen 1 & -- & Materials and supplies needed for the construction and installation of one (1) LAGO water Cherenkov detector, excluding the PMT, the LAGO digitizer board and the Nexys 2 development board (see table 3) \\ \fi
\hline
\end{tabular}
\end{center}


Los ítems provistos por la \institution~son de su propiedad, pero serán
utilizados exclusivamente para las operaciones de la \institution~dentro del
Proyecto LAGO. Luego de la finalización de este acuerdo, la \institution~podrá
disponer de estos ítems para los proyectos que considere pertinentes. Dado que
los ítems provistos serán utilizados en forma exclusiva por los miembros LAGO
de la \institution~en sus instalaciones o sitios operados por esta, la
\institution~no realizará reclamos a la Colaboración LAGO por el deterioro o
daños que sufrieran estos ítems.

\subsubsection{Descripción de los Ítems Provistos por la Colaboración LAGO}

La Colaboración LAGO está de acuerdo en proveer a la \institution~los
siguientes componentes, materiales, software y servicios: 

\begin{center}
\begin{tabular}{|p{1.0cm}|p{1.5cm}|p{12.7cm}|}
\hline
\ifes 
Ítem & Cantidad & Descripción \\
\hline
1 & -- & Asesoramiento técnico para el montaje, operación y mantenimiento del(los) detector(es) operados por la \institution \\
\hline
2 & -- & Software LAGO para la operación y calibración del(los) detector(es) operados por la \institution \\
\hline
3 & -- & Software LAGO para el análisis y preservación de datos LAGO (ver sección \ref{data}) \\
\hline
4 & -- & Software LAGO de simulación y acceso remoto total a sistemas de computación distribuida operados por u asociados al proyecto LAGO (clusters, GRID, etc) \\
\hline
5 & -- & Acceso remoto total al repositorio central de datos LAGO para almacenamiento y curaduría de datos producidos en el(los) detector(es) operados por la \institution \\
\hline
6 & -- & Acceso completo a los datos LAGO existentes y futuros (ver sección \ref{data}) \\
\hline
7 & -- & Diseño y lista de componentes de la placa digitalizadora LAGO. Eventualmente, y sujeto a la disponibilidad de las placas o materiales en las instituciones donde son fabricadas, el Proyecto LAGO podrá proveer una (1) placa digitalizadora completa a la \institution. \\
\hline
8 & 1 & PMT Hamamatsu modelo R5912 (8"), propiedad de LAGO, en comodato, Serie: \\
\hline
9 & 1 & Placa digitalizadora LAGO completa y funcionando.\\
\hline
10 & 1 & Placa de desarrollo Nexys II, propiedad de LAGO, completa y funcionando. \\ 
\hline
\fi

\ifen
Item & Quantity & Description \\
\hline
1 & -- & Technical know-how for the deployment, operation and maintenance of the detector(s) operated by \institution \\
\hline
2 & -- & LAGO software for the detector operation and calibration \\
\hline
3 & -- & LAGO software for the analysis and preservation of the LAGO data (see section \ref{data}) \\
\hline 
4 & -- & LAGO simulation software and full remote access to distributed computational services operated by or associated to the LAGO project (clusters, GRID, etc) \\
\hline
5 & -- & Full remote access to the central LAGO data repository for storage and preservation of the data produced by the detector(s) operated by \institution \\
\hline
6 & -- & Full access to the existent and future LAGO data (see section \ref{data}) \\
\hline
7 & -- &LAGO digitizer board design and components list. Eventually, and subject to availability on the manufacturing institutions, the LAGO Project could provide one (1) complete digitizer board to the \institution \\
\hline
8 & 1 & PMT Hamamatsu model R5912 (8"), owned by the LAGO project, in comodate, Serial number: \\
\hline
9 & 1 & LAGO digitizer board, owned by the LAGO project, in comodate. \\
\hline
10 & 1 & Nexys II development FPGA board, owned by the LAGO project, in comodate. \\
\hline
\fi

\ifpt
Item & Quantity & Description \\
\hline
1 & -- & Technical know-how for the deployment, operation and maintenance of the detector(s) operated by \institution \\
\hline
2 & -- & LAGO software for the detector operation and calibration \\
\hline
3 & -- & LAGO software for the analysis and preservation of the LAGO data (see section \ref{data}) \\
\hline 
4 & -- & LAGO simulation software and full remote access to distributed computational services operated by or associated to the LAGO project (clusters, GRID, etc) \\
\hline
5 & -- & Full remote access to the central LAGO data repository for storage and preservation of the data produced by the detector(s) operated by \institution \\
\hline
6 & -- & Full access to the existent and future LAGO data (see section \ref{data}) \\
\hline
7 & -- &LAGO digitizer board design and components list. Eventually, and subject to availability on the manufacturing institutions, the LAGO Project could provide one (1) complete digitizer board to the \institution \\
\hline
\fi

\end{tabular}
\end{center}
 

La \institution~recibe estos ítems, los cuales son entregados por la
Colaboración LAGO bajo la figura de  {\it{comodato}} y serán utilizados
exclusivamente por los miembros LAGO de la \institution~para su uso específico
como parte del proyecto LAGO. Por solicitud de una de las partes o con la
finalización de  este acuerdo, estos ítems deben ser devueltos a la
Colaboración LAGO en esencia y sin deterioro (no atribuible al uso adecuado, o desgaste natural) o bien en especie, después de
acuerdo específico entre las partes.

\subsubsection{Instalación, Puesta en Marcha y Operación}

La \institution\ participará en la instalación, puesta en marcha y
operación de los equipos del proyecto LAGO y poseerá personal en los
respectivos sitios de acuerdo a este listado:

\begin{itemize}
  \ifes
    \item Sitio: ''Pasto'' (PSO) localizado en la \institution. 
  \fi
  \ifen
    \item Site: ''Pasto'' (PSO), located at the \institution. 
  \fi
  \ifpt
    \item Site: ''Pasto'' (PSO), located at the \institution. 
  \fi
\end{itemize}
%% Repeat as necessary for other subsystems in which \institution is participating


\subsubsection{Acceso a los datos}\label{data}

Todos los datos producidos por el proyecto LAGO son propiedad de la
Colaboración LAGO, tanto aquellos recolectados por los detectores asociados al
proyecto (``datos medidos''), como aquellos datos producidos por simulaciones
asociadas con el proyecto (``datos sintéticos''), de aquí en adelante referidos
en conjunto como `los datos LAGO''. A pesar de ello, la Colaboración LAGO
reconoce la autoría de los distintos conjuntos de datos que la
\institution~produzca en el marco del proyecto LAGO.

Todos los miembros LAGO de la \institution~tienen acceso total a todos los
conjuntos de datos LAGO y al software oficial desarrollado en el marco del
proyecto para el análisis de estos datos. Las partes acuerdan la provisión de
la infraestructura o los métodos necesarios para garantizar el acceso a los
datos LAGO de forma regular. Los datos producidos por los detectores LAGO
asociados a la \institution~serán tratados de acuerdo a las políticas de datos
establecidas por el Comité Científico de la Colaboración LAGO.

\subsection{Reportes}

\ifirg El \else La \fi \instrep~reportará trimestralmente todos los
desembolsos, progresos técnicos asociados y estado del proyecto en la
\institution~al Representante de \country. Los reportes de todas las
instituciones de la Colaboración LAGO serán integrados y hechos públicos a toda
la Colaboración LAGO por el Comité de Coordinación del Proyecto LAGO.

\section{Firmantes}

Las siguientes personas concurren en los términos de este Memorándum de
Entendimiento. Estos términos serán actualizados de acuerdo a la necesidad como
Enmiendas a este Memorándum.

\vspace*{2cm}
\vspace*{2cm} % comment this and uncomment includegraphics to include digital signatures
\begin{center}
\noindent\begin{tabular}{lll}
%    \includegraphics[width=0.3\columnwidth]{sign-asorey.png} & \includegraphics[width=0.3\columnwidth]{sign-asorey.png} & \includegraphics[width=0.3\columnwidth]{sign-asorey.png} \\

  \instrep                        &  \countryrep & \lagopi \\
  \ifes 
  Representante de la Institución & Representante de \country & Investigador Principal \\ 
  \institution                          & Colaboración LAGO         & Colaboración LAGO \\ 
  \fi
  \ifen 
  Institution Representative & \country Representative & Principal Investigator \\
  \institution               & LAGO Collaboration      & LAGO Collaboration \\
  \fi
  \ifpt 
  NNN                        & NNN \country            & NNN                    \\ 
  \institution               & Colaboracion LAGO       & Colaboración LAGO      \\
  \fi
\end{tabular}
\end{center}

